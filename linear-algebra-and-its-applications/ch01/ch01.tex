\documentclass[12pt]{article}
\usepackage{amsmath, amssymb, amsthm}
\usepackage{graphicx}
\usepackage{amscd}
\usepackage{enumerate}
\usepackage{color}
\usepackage{cancel}
\usepackage{tikz}
\usepackage{empheq}
\usetikzlibrary{matrix}
\usepackage[thinlines]{easytable}
\usepackage[left=2cm,top=1.5cm,right=2cm,bottom=1cm,nohead,nofoot]{geometry}

% define an "answer box" for highlighting final answer
\definecolor{anscolor}{rgb}{.87, .77, 1}

\newlength\mytemplen
\newsavebox\mytempbox

\makeatletter
\newcommand\ansbox{%
    \@ifnextchar[%]
       {\@ansbox}%
       {\@ansbox[0pt]}}

\def\@ansbox[#1]{%
    \@ifnextchar[%]
       {\@@ansbox[#1]}%
       {\@@ansbox[#1][0pt]}}

\def\@@ansbox[#1][#2]#3{
    \sbox\mytempbox{#3}%
    \mytemplen\ht\mytempbox
    \advance\mytemplen #1\relax
    \ht\mytempbox\mytemplen
    \mytemplen\dp\mytempbox
    \advance\mytemplen #2\relax
    \dp\mytempbox\mytemplen
    \colorbox{anscolor}{\hspace{1em}\usebox{\mytempbox}\hspace{1em}}}

\makeatother

% DOCUMENT BEGINNING %
\begin{document}

\begin{center}
    Chapter 1
\end{center}

\subsection*{Section 1.1}

\subsubsection*{Problem 1}
a) line

\noindent b) plane

\noindent c) $\mathbb{R}^3$

\subsubsection*{Problem 3}
Manipulate $\vec{v} - \vec{w} = \begin{bmatrix} 1 \\ 5 \end{bmatrix}$ to be 
$\vec{v} = \begin{bmatrix} 1 \\ 5 \end{bmatrix} + \vec{w}$. Then subsitute in $\vec{v}$ 
within $\vec{v} + \vec{w} = \begin{bmatrix} 5 \\ 1 \end{bmatrix}$ to yield 
$\begin{bmatrix} 1 \\ 5 \end{bmatrix} + \vec{w} + \vec{w} = \begin{bmatrix} 5 \\ 1 \end{bmatrix}$. 
Hence, $2\vec{w} = \begin{bmatrix} 4 \\ -4 \end{bmatrix}$ and 
$\vec{w} = \begin{bmatrix} 2 \\ -2 \end{bmatrix}$. Substituting back in yields
$\vec{v} = \begin{bmatrix} 3 \\ 3 \end{bmatrix}$

\subsubsection*{Problem 5}
\[ \begin{bmatrix} 1 \\ 2 \\ 3 \end{bmatrix} 
+ \begin{bmatrix} -3 \\ 1 \\ -2 \end{bmatrix} 
+ \begin{bmatrix} 2 \\ -3 \\ 1 \end{bmatrix} 
= \begin{bmatrix} 0 \\ 0 \\ 0 \end{bmatrix} \]

\[2\begin{bmatrix} 1 \\ 2 \\ 3 \end{bmatrix} 
+ 2\begin{bmatrix} -3 \\ 1 \\ -2 \end{bmatrix} 
+ \begin{bmatrix} 2 \\ -3 \\ 1 \end{bmatrix} 
= \begin{bmatrix} -2 \\ 3 \\ 1 \end{bmatrix} \]

\[\vec{w} = -\vec{u} -\vec{v} \]

\noindent Since $\vec{w}$ is a linear combination of $\vec{u}$ and $\vec{v}$, 
$\vec{w}$ reaches no part of 3-D space that a linear combination of  $\vec{u}$ and 
$\vec{v}$ (a plane) cannot reach. Hence, the three vectors only span a plane. 

\subsubsection*{Problem 9}
The three possible corners arr $(4, 4), (4, 0), (-2, 2)$

\subsubsection*{Problem 11}
The other four corners are $(1, 1, 0), (0, 1, 1), (1, 0, 1), (1, 1, 1)$. The 
center of the cube is $(0.5, 0.5, 0.5)$. The cube has 12 edges. 

\subsubsection*{Problem 13} 
a) Since there are 6 pairs of vectors that oppose each other, 
that is their sum is zero, then the sum of all the vectors must be the origin 
$(0, 0, 0)$ (the center of the clock).
~\\

\noindent 
b) Since the 8:00 no longer has an opposing vector (was 2:00) and all other 5 pairs of 
vectors still do, 8:00 is the only vector left. 
~\\

\noindent
c) Since there are 12 vectors that are evenly spaced within 360 degrees, each has an 
angle of 30 degrees in between. Let $\theta$ be the angle between 2:00 and 3:00. As was
just shown, $\theta = 30$ degrees. Hence $\vec{v} = (\cos(30), \sin(30)) =
(\frac{1}{2}, \frac{\sqrt{3}}{2})$

\newpage
\subsection*{Section 1.2}

\subsubsection*{Problem 1}
\[ \vec{u} \cdot \vec{v} = 
\begin{bmatrix} -0.6 \\ 0.8 \end{bmatrix} \cdot 
\begin{bmatrix} 4 \\ 3 \end{bmatrix} = 
(-0.6)(4) + (0.8)(3) = -2.4 + 2.4 = 0 \]

\[ \vec{u} \cdot \vec{w} = 
\begin{bmatrix} -0.6 \\ 0.8 \end{bmatrix} \cdot 
\begin{bmatrix} 1 \\ 2 \end{bmatrix} = 
(-0.6)(1) + (0.8)(2) = -0.6 + 1.6 = 1 \]

\[ \vec{u} \cdot (\vec{v} + \vec{w}) = 
\begin{bmatrix} -0.6 \\ 0.8 \end{bmatrix} \cdot 
(\begin{bmatrix} 4 \\ 3 \end{bmatrix} +
\begin{bmatrix} 1 \\ 2 \end{bmatrix}) = 
\begin{bmatrix} -0.6 \\ 0.8 \end{bmatrix} \cdot 
\begin{bmatrix} 5 \\ 5 \end{bmatrix} =
(-0.6)(5) + (0.8)(5) = -3 + 4 = 1 \]

\[ \vec{w} \cdot \vec{v} = 
\begin{bmatrix} 4 \\ 3 \end{bmatrix} \cdot 
\begin{bmatrix} 1 \\ 2 \end{bmatrix} = 
(4)(1) + (3)(2) = 4 + 6 = 10 \]

\subsubsection*{Problem 3}
\[ \hat v = \frac{\vec{v}}{ \lVert \vec{v} \rVert } 
= \frac{\vec{v}}{ \sqrt{4^2 + 3^2} }
= \frac{\vec{v}}{ \sqrt{25} }
= \frac{\begin{bmatrix} 4 \\ 3 \end{bmatrix}}{ 5 }
= \begin{bmatrix} 0.8 \\ 0.6 \end{bmatrix} \]

\[ \hat w = \frac{\vec{w}}{ \lVert \vec{w} \rVert } 
= \frac{\vec{w}}{ \sqrt{1^2 + 2^2} }
= \frac{\begin{bmatrix} 4 \\ 3 \end{bmatrix}}{ \sqrt{5} }
= \begin{bmatrix} \frac{4}{\sqrt{5}} \\ \frac{3}{\sqrt{5}} \end{bmatrix} \]

\[ \vec{a} = \begin{bmatrix} 2 \\ 4 \end{bmatrix} \]

\[ \vec{b} = \begin{bmatrix} -2 \\ 1 \end{bmatrix} \]

\[ \vec{c} = \begin{bmatrix} -3 \\ -6 \end{bmatrix} \]

\subsubsection*{Problem 5}
\[ \hat u_1 = \frac{\vec{v}}{ \lVert \vec{v} \rVert } 
= \frac{\vec{v}}{ \sqrt{1^2 + 3^2} }
= \frac{\begin{bmatrix} 4 \\ 3 \end{bmatrix}}{ \sqrt{10} }
= \begin{bmatrix} \frac{1}{\sqrt{10}} \\ \frac{3}{\sqrt{10}} \end{bmatrix} \]

\[ \hat u_2 = \frac{\vec{w}}{ \lVert \vec{w} \rVert } 
= \frac{\vec{w}}{ \sqrt{2^2 + 1^2 + 2^2} }
= \frac{\begin{bmatrix} 2 \\ 1 \\ 2 \end{bmatrix}}{ \sqrt{9} }
= \begin{bmatrix} \frac{2}{9} \\ \frac{1}{9} \\ \frac{2}{9} \end{bmatrix} \]

\[ \hat U_1 = \begin{bmatrix} \frac{-3}{\sqrt{10}} \\ \frac{1}{\sqrt{10}} \end{bmatrix}
\hat U_2 = 
\begin{bmatrix} \frac{-2}{9} \\ \frac{-1}{9} \\ \frac{-2}{9} \end{bmatrix} 
\]

\end{document}