\documentclass[12pt]{article}
\usepackage{amsmath, amssymb, amsthm}
\usepackage{graphicx}
\usepackage{amscd}
\usepackage{enumerate}
\usepackage{color}
\usepackage{cancel}
\usepackage{tikz}
\usetikzlibrary{matrix}
\usepackage[thinlines]{easytable}

\usepackage[left=2cm,top=1.5cm,right=2cm,bottom=1cm,nohead,nofoot]{geometry}

%\addtolength{\hoffset}{-.25in} \addtolength{\textwidth}{.25in}

%opening

\begin{document}
\thispagestyle{empty}
\pagestyle{empty}

\begin{center}
    Chapter 1: Divisibility and Congruence
\end{center}
\subsection*{Definitions}
\begin{enumerate}
\item Natural Numbers: $\{1, 2, 3, 4, ...\}$
\item Integers: $\{..., -3, -2, -1, 0, 1, 2, 3, ...\}$
\item Suppose $a$ and $d$ are integers. Then $d$ \emph{divides} $a$, denoted $d \vert a$, if and only if there is an integer $k$ such that $a = kd$. 
\item Suppose that $a$, $b$, and $n$ are integers, with $n > 0$. We say that $a$ and $b$ are \emph{congruent modulo $n$} if and only if $n \vert (a - b)$. We denote this relationship as $a \equiv b$ (mod $n$) and read these symbols as \emph{$a$ is congruent to $b$ modulo $n$}.
\end{enumerate}

\newpage
\subsection*{Exercises}
\noindent \textbf{1.1 Theorem.} 
\emph{Let  $a$, $b$, and $c$ be integers. 
If $a \vert b$ and $a \vert c$, then $a \vert (b+c)$.}
~\\
~\\
\textbf{Proof:}  
Suppose $a$, $b$, and $c$ to be integers such that $a \vert b$ and $a \vert c$. By definition of divides, $b = ka$ for some $k \in \mathbb{Z}$ and $c = la$ for some $l \in \mathbb{Z}$. By substitution, $b + c = ka + la = a (k + l)$. Because integers are closed under addition, then $(k + l) \in \mathbb{Z}$. Therefore, by definition of divides, $a \vert (b + c)$. \qed
~\\

\noindent \textbf{1.2 Theorem.} 
\emph{Let  $a$, $b$, and $c$ be integers. 
If $a \vert b$ and $a \vert c$, then $a \vert (b-c)$.}
~\\
~\\
\textbf{Proof:}
Suppose $a$, $b$, and $c$ to be integers such that $a \vert b$ and $a \vert c$. By definition of divides, $b = ka$ for some $k \in \mathbb{Z}$ and $c = la$ for some $l \in \mathbb{Z}$. By substitution, $b - c = ka - la = a (k - l)$. Because integers are closed under addition, then $(k - l) \in \mathbb{Z}$. Therefore, by definition of divides, $a \vert (b - c)$. \qed
~\\

\noindent \textbf{1.3 Theorem.}
\emph{Let $a$, $b$, and $c$ be integers. If $a \vert b$ and $a \vert c$, 
then $a \vert bc$.}
~\\
~\\
\textbf{Proof:}
Suppose $a$, $b$, and $c$ to be integers such that $a \vert b$ and $a \vert c$. By definition of divides, $b = ka$ for some $k \in \mathbb{Z}$ and $c = la$ for some $l \in \mathbb{Z}$. By substition, $bc = (ka) \cdot (la) = a \cdot (akl)$. Because integers are closed under multiplication, then $akl \in \mathbb{Z}$. Therefore, by definition of divides, $a \vert bc$. \qed
~\\

\noindent \textbf{1.4a Question.} 
\emph{Can you weaken the hypothesis of the previous theorem and
still prove the conclusion? }
~\\
~\\
\textbf{Answer:} Yes, $a$ only needs to divide $b$ or $c$ instead of 
$b$ and $c$. 
~\\
~\\
\textbf{1.4b Question.}
\emph{Can you keep the same hypothesis, but replace
the conclusion by the stronger conclusion that $a^2 \vert bc$ and still prove the
theorem?}
~\\
~\\
\textbf{Answer:} Yes, you can. 
~\\

\noindent \textbf{1.5 Theorem.} 
\emph{Let $a$, $b$, and $c$ be integers. If $a \vert b$ and $a \vert c$, 
then $a^2 \vert bc$.}
~\\
~\\
\textbf{Proof:}
Suppose $a$, $b$, and $c$ to be integers such that $a \vert b$ and $a \vert c$. By definition of divides, $b = ka$ for some $k \in \mathbb{Z}$ and $c = la$ for some $l \in \mathbb{Z}$. By substition, $bc = (ka) \cdot (la) = a^2 \cdot (kl)$. Because integers are closed under multiplication, then $kl \in \mathbb{Z}$. Therefore, by definition of divides, $a^2 \vert bc$. \qed
~\\

\noindent \textbf{1.6 Theorem.} 
\emph{Let $a$, $b$, and $c$ be integers. If $a \vert b$, 
then $a \vert bc$.}
~\\
~\\
\textbf{Proof:}
Suppose $a$, $b$, and $c$ to be integers such that $a \vert b$. By definition of divides, $b = ka$ for some $k \in \mathbb{Z}$. By substitution, $bc = (ka) \cdot c = a \cdot (kc)$. Because integers are closed under multiplication, then $kc \in \mathbb{Z}$. Therefore, by definition of divides, $a \vert bc$. \qed
~\\

\newpage
~\\

\noindent \textbf{1.7 Exercise.} \emph{Answer each of the following questions, 
and prove that your answer is correct.}
~\\
\emph{1. Is} 45 $\equiv$ 9 (mod 4)\emph{?} Yes, $4 \vert (45 - 9)$. 
~\\
\emph{2. Is} 37 $\equiv$ 2 (mod 5)\emph{?} Yes, $5 \vert (37 - 2)$. 
~\\
\emph{3. Is} 37 $\equiv$ 3 (mod 5)\emph{?} No, $5 \not \vert (37 - 3)$. 
~\\
\emph{4. Is} 37 $\equiv$ -3 (mod 5)\emph{?} Yes, $5 \vert (37 - (-3))$. 
~\\

\noindent \textbf{1.8 Exercise.} \emph{For each of the following congruences, 
characterize all the integers m that satisfy that congruence.}
~\\
\emph{1.} $m \equiv 0$ (mod 3). \; $m = 3k \; \; \; k \in \mathbb{Z}$
~\\
\emph{2.} $m \equiv 1$ (mod 3). \; $m = 3k + 1 \; \; \; k \in \mathbb{Z}$
~\\
\emph{3.} $m \equiv 2$ (mod 3). \; $m = 3k + 2 \; \; \; k \in \mathbb{Z}$
~\\
\emph{4.} $m \equiv 3$ (mod 3). \; $m = 3k \; \; \; k \in \mathbb{Z}$
~\\
\emph{5.} $m \equiv 4$ (mod 3). \; $m = 3k + 1 \; \; \; k \in \mathbb{Z}$
~\\

\noindent \textbf{1.9 Theorem.} \emph{Let $a$ and $n$ be integers with $n > 0$.
Then $a \equiv a$} (mod $n$). 
~\\
~\\
\textbf{Proof:} 
Suppose $a$ and $n$ be integers with $n > 0$. By algebra, $a - a = 0 = n \cdot 0$. So, by definition of divides, $n \vert (a - a)$. Therefore, by definition of congruence, $a \equiv a$ (mod $n$)
\qed
~\\

\noindent \textbf{1.10 Theorem.} \emph{Let $a$, $b$, and $n$ be integers with $n>0$. 
If $a \equiv b$} (mod $n$)\emph{, then $b \equiv a$} (mod $n$).
~\\
~\\
\textbf{Proof:} 
Suppose $a$ and $n$ be integers with $n > 0$ such that $a \equiv b$ (mod $n$). By definition of congruence, $n \vert (a - b)$. By definition of divides, $a - b = nk$ where $k \in \mathbb{Z}$. Multiplying both sides by $-1$, $b - a \equiv n(-k)$. Since integers are closed under multiplication, $(-k) \in \mathbb{Z}$. Hence, $n \vert (b-a)$ by definition of divides. Therefore, $b \equiv a$ (mod $n$). 
 \qed
~\\

\noindent \textbf{1.11 Theorem.} \emph{Let $a$, $b$, $c$, and $n$ be integers with $n>0$. If $a \equiv b$ (}mod\emph{ $n$) and $b \equiv c$ (}mod\emph{ $n$), then $a \equiv c$ (}mod\emph{ $n$).}
~\\
~\\
\textbf{Proof:} 
Suppose $a$, $b$, $c$, and $n$ to be integers with $n>0$ such that $a \equiv b$ (mod $n$) and $b \equiv c$ (mod $n$). By definition of congruence, $n \vert (a - b)$ and $n \vert (b - c)$. By definition of divides, $a - b = kn$ where $k \in \mathbb{Z}$ and $b - c = ln$ where $l \in \mathbb{Z}$. By substitution, $a - c = (a - b) + (b - c) = kn + ln = n( k + l)$. Since integers are closed under addition, $(k + l) \in \mathbb{Z}$. Thus, by definition of divides, $n \vert (a - c)$. Therefore, by definition of congruence, $a \equiv c$ (mod $n$). \qed

\end{document}