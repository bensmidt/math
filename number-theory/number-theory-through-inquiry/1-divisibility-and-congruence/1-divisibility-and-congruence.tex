\documentclass[12pt]{article}
\usepackage{amsmath, amssymb, amsthm}
\usepackage{graphicx}
\usepackage{amscd}
\usepackage{enumerate}
\usepackage{color}
\usepackage{cancel}
\usepackage{tikz}
\usetikzlibrary{matrix}
\usepackage[thinlines]{easytable}

\usepackage[left=2cm,top=0.5cm,right=3cm,bottom=1cm,nohead,nofoot]{geometry}

%\addtolength{\hoffset}{-.25in} \addtolength{\textwidth}{.25in}


\newcommand{\ds}{\displaystyle}


\newcommand{\al}{\alpha}
\newcommand{\be}{\beta}

\newcommand{\lcm}{\text{lcm}}
\newcommand{\ord}{\text{ord}}
\newcommand{\modulo}[1]{\text{ (mod $#1$)}}
\newcommand{\pa}[1]{\left(#1\right)}


\newcommand{\bN}{\mathbb{N}}
\newcommand{\bZ}{\mathbb{Z}}
\newcommand{\bQ}{\mathbb{Q}}
\newcommand{\bR}{\mathbb{R}}
\newcommand{\bC}{\mathbb{C}}

\newcommand{\cF}{\mathcal{F}}
\newcommand{\cG}{\mathcal{G}}
\newcommand{\cH}{\mathcal{H}}
\newcommand{\cI}{\mathcal{I}}
\newcommand{\cJ}{\mathcal{J}}
\newcommand{\cK}{\mathcal{K}}
\newcommand{\cL}{\mathcal{L}}
\newcommand{\cP}{\mathcal{P}}
\newcommand{\cQ}{\mathcal{Q}}
\newcommand{\cR}{\mathcal{R}}
\newcommand{\cS}{\mathcal{S}}


\newcommand{\Array}[1]{\begin{array}{lllllllllllllllll}#1\end{array}}

\newcommand{\thm}[1]{\noindent {\bf Theorem #1.}}

\newcommand{\defn}[1]{\noindent {\bf Definition.}}

\newcommand{\lem}[1]{\noindent {\bf Lemma #1.}}

\newcommand{\cor}[1]{\noindent {\bf Corollary #1.}}

\newcommand{\ex}[1]{\noindent {\bf Exercise #1.}}

\newcommand{\qn}[1]{\noindent {\bf Question #1.}}

\newcommand{\prob}[1]{\noindent {\bf Problem #1.}}




%opening

\begin{document}
\thispagestyle{empty}
\pagestyle{empty}

\begin{center}
    Chapter 1: Divisibility and Congruence
\end{center}
~\\
~\\
\noindent \textbf{1.1 Theorem.} 
\emph{Let  $a$, $b$, and $c$ be integers. 
If $a \vert b$ and $a \vert c$, then $a \vert (b+c)$.}
~\\
~\\
\textbf{Proof:}  
By definition of divides \qed
~\\
~\\


\noindent \textbf{1.2 Theorem.} 
\emph{Let  $a$, $b$, and $c$ be integers. 
If $a \vert b$ and $a \vert c$, then $a \vert (b-c)$.}
~\\
~\\
\textbf{Proof:}  \qed
~\\
~\\

\noindent \textbf{1.3 Theorem.}
\emph{Let $a$, $b$, and $c$ be integers. If $a \vert b$ and $a \vert c$, 
then $a \vert bc$.}
~\\
~\\
\textbf{Proof:}
 \qed
~\\
~\\

\noindent \textbf{1.4a Question.} 
\emph{Can you weaken the hypothesis of the previous theorem and
still prove the conclusion? }
~\\
~\\
\textbf{Answer:} Yes, $a$ only needs to divide $b$ or $c$ instead of 
$b$ and $c$. 
~\\
~\\
\textbf{1.4b Question.}
\emph{Can you keep the same hypothesis, but replace
the conclusion by the stronger conclusion that $a^2 \vert bc$ and still prove the
theorem?}
~\\
~\\
\textbf{Answer:} Yes, you can. 
~\\
~\\

\newpage
~\\

\noindent \textbf{1.5 Theorem.} 
\emph{Let $a$, $b$, and $c$ be integers. If $a \vert b$ and $a \vert c$, 
then $a^2 \vert bc$.}
~\\
~\\
\textbf{Proof:}
\qed
~\\
~\\

\noindent \textbf{1.6 Theorem.} 
\emph{Let $a$, $b$, and $c$ be integers. If $a \vert b$, 
then $a \vert bc$.}
~\\
~\\
\textbf{Proof:}
 \qed
~\\
~\\

\noindent \textbf{1.7 Exercise.} \emph{Answer each of the following questions, 
and prove that your answer is correct.}
~\\
\emph{1. Is} 45 $\equiv$ 9 (mod 4)\emph{?} 
~\\
\emph{2. Is} 37 $\equiv$ 2 (mod 5)\emph{?} 
~\\
\emph{3. Is} 37 $\equiv$ 3 (mod 5)\emph{?} 
~\\
\emph{4. Is} 37 $\equiv$ -3 (mod 5)\emph{?}
~\\

\noindent \textbf{1.8 Exercise.} \emph{For each of the following congruences, 
characterize all the integers m that satisfy that congruence.}
~\\
\emph{1.} $m \equiv 0$ (mod 3). \; $m = 3k \; \; \; k \in \mathbb{Z}$
~\\
\emph{2.} $m \equiv 1$ (mod 3). \; $m = 3k + 1 \; \; \; k \in \mathbb{Z}$
~\\
\emph{3.} $m \equiv 2$ (mod 3). \; $m = 3k + 2 \; \; \; k \in \mathbb{Z}$
~\\
\emph{4.} $m \equiv 3$ (mod 3). \; $m = 3k \; \; \; k \in \mathbb{Z}$
~\\
\emph{5.} $m \equiv 4$ (mod 3). \; $m = 3k + 1 \; \; \; k \in \mathbb{Z}$
~\\
~\\

\noindent \textbf{1.9 Theorem.} \emph{Let $a$ and $n$ be integers with $n > 0$.
Then $a \equiv a$} (mod $n$). 
~\\
~\\
\textbf{Proof:} 
\qed
\newpage
~\\

\noindent \textbf{1.10 Theorem.} \emph{Let $a$, $b$, and $n$ be integers with $n>0$. 
If $a \equiv b$} (mod $n$)\emph{, then $b \equiv a$} (mod $n$).
~\\
~\\
\textbf{Proof:} 
 \qed
~\\
~\\

\noindent \textbf{1.11 Theorem.} \emph{Let $a$, $b$, $c$, and $n$ be integers with $n>0$. 
If $a \equiv b$ (}mod\emph{ $n$) and $b \equiv c$ (}mod\emph{ $n$), then $a \equiv c$ 
(}mod\emph{ $n$).}
~\\
~\\
\textbf{Proof:}  \qed

\end{document}